% Simple Beamer Template
\documentclass{beamer}
\beamertemplatenavigationsymbolsempty
\setbeamertemplate{footline}[page number]
\usepackage{pdfpages}
%\newcommand{\T}{^{\mathsf{T}}}
%\usepackage{array,tabularx,tabulary,booktabs}
%\usepackage{longtable}
%\usepackage{multirow}

\title{Distribution of model parameters and causality}
\author{IA pour la Science}
\date{January $23^\text{th}$, 2026}

\begin{document}
\begin{frame}
  \titlepage
\end{frame}
%=============================================
\begin{frame}{Distribution of model parameters and causality}
The research focuses on the causal inference for the model selection in machine learning.
To select a model we need to estimate the distribution of model parameters.

PyTorch provides a well-developed automated differentiation framework to estimate the parameters of the model.
The nearest goal is to propose a method to build Causal Structural Model through the distribution of model parameters and use it for model selection purposes.

Below:
\begin{enumerate}
    \item Causal machine and Control Problem
    \item Clarification of the CI for time series
	\item Applications of the Causal Discovery
\end{enumerate}
\end{frame}
%=============================================
\begin{frame}{Causal Inference and Control Problem}
	\includegraphics[width=\textwidth]{tmp/fig_control_pid}
\end{frame}
%=============================================
\begin{frame}{Causal Chambers\footnote{https://www.nature.com/articles/s42256-024-00964-x}}
	\includegraphics[width=\textwidth]{tmp/causal_chambers_1}
\end{frame}
%=============================================
\begin{frame}{Causal Chambers (2)}
	\includegraphics[width=\textwidth]{tmp/causal_chambers_2}
\end{frame}
%=============================================
\begin{frame}{Causal Chambers (3)}
	\includegraphics[width=\textwidth]{tmp/causal_chambers_3}
\end{frame}
%=============================================
\begin{frame}{Time-delay embedding in Singula Spectrum Analysis}
	\includegraphics[width=\textwidth]{tmp/fig_td_embedding_ssa}
\end{frame}
%=============================================
\begin{frame}{Model of the phase trajectory in the state space}
	\includegraphics[width=\textwidth]{tmp/fig_phase_trajectory}
\end{frame}
%=============================================
\begin{frame}{Linear model for time series forecasting}
	\includegraphics[width=\textwidth]{tmp/fig_linear_forecasting}
\end{frame}
%=============================================
\begin{frame}{Conrol problem for time series forecasting\footnote{https://doi.org/10.1038/s41467-021-21554-0}}
	\includegraphics[width=\textwidth]{tmp/fig_nature}
\end{frame}
%=============================================
\begin{frame}{Covariance matrix of forecasting model parameters}
	Estimate parameters $\mathbf{w}(\tau) = (\mathbf{X}^\mathsf{T}\mathbf{X})^{-1}\mathbf{X}^\mathsf{T}\mathbf{y}$,
	then calculate the sample $s(\tau)=\mathbf{w}^\mathsf{T}(\tau)\mathbf{X}_{m+1}$
	for each $\tau$ of the next $m + 1$-th period.\\
	\centering{\includegraphics[width=0.6\textwidth]{tmp/fig_linear_covariance}}\\
	The color shows the value of a parameter for each hour.
\end{frame}
%=============================================
\begin{frame}{Probabilistic model selection}
	\includegraphics[width=\textwidth]{tmp/fig_bayesian_inference}
\end{frame}
%=============================================
\begin{frame}{ Empirical distribution of model parameters}
	\includegraphics[width=1.1\textwidth]{tmp/fig_bayesian_sampling}
\end{frame}
%=============================================
\begin{frame}{Evidence of the model}
	\includegraphics[width=\textwidth]{tmp/fig_bayesian_evidence}
\end{frame}
%=============================================
%\begin{frame}{Neural network model selection}
%	\includegraphics[width=\textwidth]{tmp/fig_bayesian_nn}
%\end{frame}
%=============================================
\begin{frame}{Do-operations and tests}
	\includegraphics[width=0.3\textwidth]{tmp/fig_do_sin}
\end{frame}
%=============================================
\begin{frame}{High variance and high co-variance in time series}
	Dynamic graph reflects dependencies between the time series.
	\includegraphics[width=\textwidth]{tmp/fig_pairwise_reduction}
	To reconstruct the dependencies
	\begin{enumerate}
		\item Reconstruct phase trajectories through time-delay embedding.
		\item Define distance between points of the phase trajectories.
		\item Make low-rank decomposition, prune the dependency graph.
		\item Reconstruct time series.
	\end{enumerate}
\end{frame}
%=============================================
\begin{frame}{Causal discovery for syblolic regression modeling}
	\includegraphics[width=\textwidth]{tmp/fig_symbolic_regression}
\end{frame}
%=============================================
\includepdf[pages=-]{tmp/structure_learning.pdf}
%=============================================
\includepdf[pages=-]{tmp/partial_order.pdf}
\end{document}